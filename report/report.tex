\documentclass[12pt,a4paper]{article}
\usepackage[utf8]{inputenc}
\usepackage{graphicx}
\usepackage{booktabs}
\usepackage{amsmath}
\usepackage{geometry}
\usepackage{float}
\usepackage{hyperref}
\geometry{margin=1in}

\title{Residential Energy Consumption Analytics Project}
\author{Bhavesh Gaikwad}
\date{\today}

\begin{document}
\maketitle

\section{Introduction}

With the evolution of smart grids and increasing electricity costs, there is a growing need to analyze household energy consumption patterns. This project leverages Python (NumPy, Pandas, Matplotlib) to process and interpret smart meter data, investigate the impact of time-of-use pricing, and derive actionable insights for consumption optimization.

\section{Problem Statement}

Analyze 36,000 hourly records from 50 households over 30 days. Tasks include:
\begin{itemize}
\item Time-series data preprocessing
\item Feature engineering for pricing, occupancy, and efficiency metrics
\item Statistical and comparative analysis
\item Visualization and business insights
\end{itemize}

\section{Methodology}

Data pipeline:
\begin{itemize}
    \item Imputed missing values for consumption and temperature
    \item Engineered features: hour\_of\_day, peak\_hours, cost\_inr, consumption\_per\_degree
    \item Performed time-series aggregation and rolling mean
    \item Grouped by occupancy and day-of-week
    \item Quantified savings and CO$_2$ impact
\end{itemize}

\section{Results}

\subsection{Hourly Consumption Pattern}
\begin{figure}[H]
    \centering
    \includegraphics[width=0.9\textwidth]{../figs/hourly_consumption_pattern.png}
    \caption{Average hourly consumption pattern with peak hours highlighted in red.}
\end{figure}

\subsection{Heatmap: Consumption by Day and Hour}
\begin{figure}[H]
    \centering
    \includegraphics[width=0.9\textwidth]{../figs/consumption_heatmap.png}
    \caption{Heatmap of average consumption (kWh) by day of week and hour of day.}
\end{figure}

\subsection{Temperature vs Consumption Scatter}
\begin{figure}[H]
    \centering
    \includegraphics[width=0.9\textwidth]{../figs/temp_vs_consumption_scatter.png}
    \caption{Scatter plot with regression line showing relationship between temperature and energy consumption.}
\end{figure}

\subsection{Weekday vs Weekend Box Plot}
\begin{figure}[H]
    \centering
    \includegraphics[width=0.9\textwidth]{../figs/weekday_vs_weekend_boxplot.png}
    \caption{Comparison of consumption distributions for weekdays and weekends.}
\end{figure}

\section{Business Insights and Recommendations}
\begin{itemize}
    \item \textbf{Peak Hour Utilization}: Households consume 40\%+ more energy during evenings; shifting 30\% peak usage to off-peak hours can save over 15\% on monthly bills.
    \item \textbf{Efficiency Outliers}: Top 5 high-consumption households represent significant share of total costs---targeted recommendations yield greatest impact.
    \item \textbf{Temperature Correlation}: Regression analysis shows increased consumption below 10$^\circ$C and above 28$^\circ$C; thermal insulation and appliance scheduling are advised.
    \item \textbf{Environmental Benefits}: With carbon intensity of 0.82 kg/kWh, optimizing consumption can reduce household CO$_2$ emissions by thousands of kg annually.
\end{itemize}

\section{Conclusion}

This analysis demonstrates a robust workflow for extracting insights from smart meter data, highlighting both cost savings and sustainability opportunities. The codebase (found in \texttt{code/main.py}) and dataset (\texttt{data/energy\_data.csv}) provide a reusable template for similar projects in utility analytics, smart home technology, and energy efficiency research.

\end{document}
